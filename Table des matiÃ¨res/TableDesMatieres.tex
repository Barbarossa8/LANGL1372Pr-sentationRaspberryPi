\documentclass[10pt,a4paper]{article}
\usepackage{inputenc}
\usepackage[T1]{fontenc}
\usepackage{fontspec}
\usepackage{graphicx}
\usepackage{lmodern}
\usepackage{vmargin}
\usepackage{tabularx}
\setlength{\parindent}{0mm}

\begin{document}

\titlepage{
	\today \vspace{7cm}
	\begin{flushright}\sf\Huge
	{\bfseries Oral presentation} \\[2mm]
	{\bfseries Raspberry Pi} \\[3mm]
	{\huge LANGL1372}
	\end{flushright}
	\ \\[9cm]
	\textbf{Last group} \\
	Denauw Antoine\\
	Decarvalho Borges Antonio
}

\newpage


\section{Summary}

\subsection{The Raspberry Pi Foundation}

\paragraph{}The Raspberry Pi Foundation is a charity founded in 2009 to promote the study of basic computer science in schools, and is responsible for developing a single-board computer called the Raspberry Pi, UK's best-selling PC of all time.

\paragraph{}The foundation is supported by the University of Cambridge Computer Laboratory and Broadcom. It's aim is to « promote the study of computer science and related topics, especially at school level, and to put the fun back into learning computing ».

\subsection{The Raspberry Pi}

\paragraph{}The Raspberry Pi is a series of credit card-sized single-board computers developed in England, United Kingdom by the Raspberry Pi Foundation with the intent to promote the teaching of basic computer science in schools and developing countries.

\paragraph{}Several generations of Raspberry Pi's have been released. The first (Pi 1) was released in February 2012 in basic model A and a higher specification model B. A+ and B+ models were released a year later. Raspberry Pi 2 model B was released in February 2015 and Raspberry Pi 3 model B in February 2016. These boards are priced between 20\$ and 35\$. A cut down compute model was released in April 2014 and a Pi Zero with smaller footprint and limited IO (Input/Output) capabilities released in November 2015 for 5\$ !!!

\newpage

\tableofcontents

\newpage

\section{The Raspberry Pi Foundation}

\subsection{Introduction}

\subsection{History}

\subsubsection{Co-founder}

\subsubsection{Collaboration}

\subsection{Why the Raspberry Pi}

\subsubsection{The idea behind}

\subsubsection{Interaction with computers}

\subsection{Logo}

\section{The Raspberry Pi}

\subsection{Introduction}

\subsubsection{What is it ?}

\subsubsection{Where does he come from ?}

\subsubsection{Feature}

\subsubsection{Operating system}

\subsection{Hardware}

\subsection{Overclocking}

\subsubsection{What is it ?}

\subsubsection{Why should we overclock ?}

\subsubsection{The risk behind the overclocking}

\subsubsection{Heat sink or not ?}

\subsection{Networking}

\subsection{Software}

\subsubsection{OS}

\subsection{Alternatives}

\subsubsection{BeagleBoard}

\subsubsection{Arduino ?}

\section{Projects}

\newpage

\section{Vocabulary}

\begin{itemize}
\item[•] Son objectif $\rightarrow$ It's aim
\item[•] Un architecte de système sur puce$\rightarrow$ a system-on-chip architect
\item[•] Abordable $\rightarrow$ Affordable
\item[•] Concerner $\rightarrow$ To concern
\item[•] D'année en année $\rightarrow$ Year-on-year
\item[•] Appliquer $\rightarrow$ Applying
\item[•] Une série d'ordinateur de bord unique au format carte de crédit $\rightarrow$ A series of credit card-sized single-board computers
\item[•] Pays en voie de développement $\rightarrow$ Developing countries
\item[•] Entrée / sortie $\rightarrow$ Input/Output (IO) (GPIO)
\item[•] Grossièrement $\rightarrow$ Roughly
\item[•] Dissipateur thermique $\rightarrow$ Heatsink
\end{itemize}

\section{Personnal opinion}

\paragraph{}

\paragraph{}

\section{References}

\begin{itemize}
\item https://www.raspberrypi.org/jam/
\item https://www.raspberrypi.org/downloads/noobs/
\item http://blog.mcmelectronics.com
\item https://en.wikipedia.org/wiki/Raspberry\_Pi\_Foundation
\item http://www.crim.cam.ac.uk/
\item https://en.wikipedia.org/wiki/Broadcom
\item http://www.theraspberrypiguy.com/call-for-questions-interview-with-eben-upton/
\item https://en.wikipedia.org/wiki/Buckminsterfullerene
\item https://en.wikipedia.org/wiki/Raspberry\_Pi
\item https://www.raspberrypi.org/education-fund/

\end{itemize}













\end{document}